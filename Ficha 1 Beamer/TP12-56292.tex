\documentclass[11pt]{beamer}
\usepackage[utf8]{inputenc}
\usepackage[portuguese]{babel}
\usepackage{color}
\usepackage{ulem}
\usepackage[utf8]{inputenc}

\author{Jo\~ao António-dos-Santos}
\date{17 de novembro de 2014}
\usetheme{CambridgeUS}
\title{A vida do Suricata}
\begin{document}

\begin{frame}
\maketitle
\end{frame}

\begin{frame}{Sobre o Suricata I}
O \textbf{suricata}, também chamado de \textbf{suricato} ou \textbf{suricate} (\textit{Suricata
suricatta) é um pequeno mamífero da familia Herpestidae, nativo do
deserto do Kalahari. Estes animais têm cerca de meio metro de
comprimento (incluindo a cauda), em média 730 gramas de peso, e
pelagem acastanhada.}

\textit{Têm garras afiadas nas patas, que lhes permitem escavar a superfície do
ch\~ao e dentes afiados para penetrar nas carapaças quitinosas das suas
presas. \textcolor{red} {Outra característica distinta é a sua capacidade de se elevarem nas patas traseiras, utilizando a cauda como terceiro apoio}.}
\end{frame}

\begin{frame}{Sobre o Suricata II}
\textit {Alimenta-se principalmente de insetos (cerca de 82 por cento) \%: 
\begin{itemize}
\item larvas de escaravelhos e de borboletas;
\item também ingerem milípedes,
\item aranhas,
\end{itemize}
...mas também de:
\begin{itemize}
\item escorpi\~oes,
\item pequenos vertebrados (répteis, anfíbios e aves), ovos 
\item e matéria vegetal.
\end{itemize}
\uline {S\~ao relativamente imunes ao veneno} das najas e dos
escorpi\~oes, sendo estes, inclusive, um dos alimentos que mais apreciam.}
\end{frame}

\begin{frame}{Onde avistar suricatas no habitat selvagem?}
Existem vários parques nacionais em África onde é possível avistar e até interagir com suricatas no seu habitat selvagem. No entanto existe uma
regra de ouro: os suricatas n\~ao gostam de chuva, por isso prefira dias
solarengos. Em baixo apresenta-se uma lista de parques ordenada por
número de suricatas por $Km^2$:
\vspace{1cm}

\begin{description}
\item[primeiro]Parque Kgalagadi, África do Sul, Botswana
\item[segundo]Parque nacional Karoo, África do Sul
\item[terceiro]Reserva do vale mágico do Suricata, África do Sul
\item[quarto]Parque nacional Iona, Angola
\end{description}
\end{frame}

\begin{frame} {Subespécies}

Existem atualmente três subespécies de Suricata:
\begin{itemize}
\item Suricata suricatta siricata;
\item Suricata suricatta iona;
\item Suricata suricatta majoriae.
\end{itemize}
Os indivíduos de cada subespécie apresentam características distintas
como se pode ver na tabela 2.1 e na figura 2.1.
\vspace{5mm}

\begin{description}
\item[Tabela:]\textbf{características diferenciadoras entre subespécies de Suricata - feita com framebox}
\end{description}


\begin{table}[]
\framebox{$
\begin{tabular}{c|c|c|c}
\multicolumn{1}{l|}{\textbf{}} & \textbf{suricata} & \textbf{iona}      & \textbf{majoriae} \\ \hline
\textbf{côr do pelo}           & beje amarelado    & castanho amarelado & castanho escuro   \\ \hline
\textbf{tamanho}               & 29 cm             & 25 cm              & 34 cm             \\ \hline
\textbf{peso}                  & 731g              & 698g               & 799g             
\end{tabular}$}
\end{table}
\end{frame}

\end{document}