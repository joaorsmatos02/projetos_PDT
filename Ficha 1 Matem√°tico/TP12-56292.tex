\documentclass[12pt, fleqn]{article}
\usepackage[utf8]{inputenc}
\usepackage[portuguese]{babel}
\usepackage[bottom]{footmisc}
\usepackage[utf8]{inputenc}
\title{Exercícios de PDT \\Folha 1 - Modo Matemático}
\date{}
\usepackage{amsmath}
\usepackage{titling}
\begin{document}

\maketitle

\noindent 1.
\[
a^{2+2}
\] 
2.
\\
\\ \indent Seja $f$ a funç\~ao definida por $f(x) = 3x+7$, e seja $a$ um número
\\ \indent positivo real.
\\
\\
\\3.
$$ds^2=dx_1^2+dx_2^2+dx_3^2-c^2dt^2$$
4.
\\

A funç\~ao $f$ é dada por 
$$f(x)=2x+\frac{x-7}{x^2+4}$$

para todos os reais x.
\\
\\5.
$$f(x,y,z)=3y^2z\left(3+\frac{7x+5}{1+y^2}\right)$$
6.
\\
\\ \indent As raízes de uma polinómio quadrático $ax^2 + bx + c$ com $a \neq 0$ s\~ao
\\ \indent dadas pela fórmula $$\frac{-b \pm \sqrt{b^2-4ac}}{2a}$$
7.
\[
f(x_1,x_2,...,x_n)=x_1^2+x_2^2+...+x_n^2
\]
8.$$\frac{1-x^{n+1}}{1-x}=1+x+x^2+...+x^n$$
9.$$\sum_{k=1}^{n}k^2=\frac{1}{2}n(n+1)$$
10.$$\int_{a}^{b}f(x)dx.$$
11.
\\
\\
$\int_{-N}^{N}e^xdx$
\\
\\
12.$$\int_{0}^{1}\int_{0}^{1}x^2y^2dx\  dy$$
13.$$\int_{0}^{+\infty}x^ne^{-x}dx=n!$$

\end{document}