\documentclass[12pt, a4paper]{article}
\usepackage[utf8]{inputenc}
\usepackage[portuguese]{babel}
\usepackage{color}
\usepackage{amsfonts}
\usepackage{graphicx}
\usepackage{ulem}
\usepackage[utf8]{inputenc}
\usepackage{enumitem}
\usepackage{amsmath}
\date{}
\title{PDT \\Ficha 2 - Modo Matemático}
\begin{document}

\maketitle
\section{Símbolos Matemáticos}
\begin{enumerate} [label= (\alph*)]
\item Se $L(x)$ representar "$x$ tem cabelo louro", ent\~ao a frase (1) escreve-se  simbolicamente $\forall x, L(x).$ A sua negação, que poderíamos coloquialmente redigir "nem todas as pessoas t\^em cabelo louro", escreve-se $\neg \forall, L(x)$, e é logicamente equivalente a (2), que se escreve $\exists x, \neg L(x).$
\item Se $A \subseteq B $ e $B \subseteq A$, então $A$ e $B$ t\^em exatamente os mesmos elementos e, portanto, $A = B$.
\item Sejam $A = 2, 3, 5, 7, 9, B = 1, 3, 5, 7, 9, C = \mathbb{N} , D = \{x \in \mathbb{Z}\ |\ x < 5\}$
\item $\lim\limits_{x\to\infty} $exp$(-x) = 0$ 
\item Em trigonometria, a relaç\~ao básica entre o seno e o coseno é conhecida como $Identidade$ $Trigonom \acute{e}trica$ $Fundamental$: ${\cos}^2\theta + {\sin}^2\theta = 1$
\end{enumerate}

\section{Equaç\~oes}
\begin{enumerate} [label= (\alph*)]
\item
\begin{equation}
\begin{split}
 x^2 + z^3& = \sqrt{2+3y} \\
 x\div 5& = z^{x+2\pi}
\end{split}
\end{equation}
\item
\begin{align}
x^2 + z^3& = \sqrt{2+3y}\\
x\div 5& = z^{x+2\pi}
\end{align}
\item\begin{align}
X_a& = \sqrt{x+y}&
  X_b& = \pi + 3&
    X_c = 2 + x^y&\\
Z_{ax}& = x^3 + 7&
  Z_{ay}& = \sqrt{x^4+5Y}+ 29y&
    Z_{Y_{a}} = x - 2 - y&\\
Z_{ax}& = 10&
  Z_{ay}& = 35y - 2&
    Z_{az} = 2 + y& 
\end{align}
\item
\begin{equation}
f(x) =
 \begin{cases}
  1 &\text{se } x \geq 2,\\
   0  &\text{se } 1<x<2,\\
    -1 &\text{se } x \leq 1.
 \end{cases}
\end{equation}
\end{enumerate}


\end{document}