\documentclass[11pt, a4paper]{report}
\usepackage[utf8]{inputenc}
\usepackage[portuguese]{babel}
\usepackage{color}
\usepackage[table,xcdraw]{xcolor}
\usepackage{graphicx}
\usepackage{ulem}
\usepackage[utf8]{inputenc}
\title{A vida do Suricata}
\author{João António-dos-Santos}
\date{\today}
\begin{document}

\maketitle
\chapter{Introdução}
\section{Sobre o Suricata}
O \textbf{suricata}, também chamado de \textbf{suricato} ou \textbf{suricate} (\textit{Suricata suricatta}) é um pequeno mamífero da família \textit{Herpestidae}, nativo do deserto
do Kalahari. Estes animais têm cerca de meio metro de comprimento
(incluindo a cauda), em média 730 gramas de peso, e pelagem acastanhada. Têm garras afiadas nas patas, que lhes permitem escavar a superfície do ch\~ao e dentes afiados para penetrar nas carapaças quitinosas das suas presas. \textcolor{red} {Outra característica distinta é a sua capacidade de
se elevarem nas patas traseiras, utilizando a cauda como terceiro apoio.}
\section{Características gerais}
\subsection{Alimentação}
Alimenta-se principalmente de insetos (cerca de 82\%):
\begin{itemize}
\item Larvas de escaravelhos e de borboletas;
\item milípedes;
\item aranhas.
\end{itemize}
...mas também de:
\begin{itemize}
\item escorpi\~oes;
\item pequenos vertebrados (répteis, anfíbios e aves);
\item ovos;
\item matéria vegetal.
\end{itemize}
\uline {S\~ao relativamente imunes ao veneno} das najas e dos
escorpi\~oes, sendo estes, inclusive, um dos alimentos que mais apreciam.

\chapter{Desenvolvimento}
\section{Onde avistar suricatas no habitat selvagem?}
Existem vários parques nacionais em Africa onde é possível avistar e
até interagir com suricatas no seu habitat selvagem. No entanto, existe
uma regra de ouro: os suricatas n\~ao gostam de chuva, por isso prefira
dias solarengos.

Em baixo apresenta-se uma lista de parques ordenada por número de
suricatas por $Km^2$:
\begin{enumerate}
\item Parque “Kgalagadi”, \textit{África do Sul / Botswana}
\item Parque nacional “Karoo”, \textit{África do Sul}
\item Reserva do vale mágico do Suricata, \textit{África do Sul}
\item Parque nacional Iona, \textit{Angola}
\end{enumerate}

ou
\begin{description}
\item \textbf{parque1} Parque “Kgalagadi”, \textit{África do Sul / Botswana}
\item \textbf{parque2} Parque nacional “Karoo”, \textit{África do Sul}
\item \textbf{parque3} Reserva do vale mágico do Suricata, \textit{África do Sul}
\item \textbf{parque4} Parque nacional Iona, \textit{Angola}
\end{description}
\section{Subespécies}
Existem atualmente três subespécies de Suricata:

\begin{itemize}
\item \textit{Suricata suricatta siricata;}
\item \textit{Suricata suricatta iona;}
\item \textit{Suricata suricatta majoriae.}
\end{itemize}

Os indivíduos de cada subspécie apresentam características distintas
como se pode ver na tabela 2.1.
\begin{table}[h]
\begin{tabular}{|c|c|c|c|}
\hline
\rowcolor[HTML]{9B9B9B} 
{\color[HTML]{343434} }                     & {\color[HTML]{343434} \textbf{siricata}} & {\color[HTML]{343434} \textbf{iona}}      & {\color[HTML]{343434} \textbf{majoriae}} \\ \hline
{\color[HTML]{343434} \textbf{côr do pelo}} & {\color[HTML]{343434} beje amarelado}    & {\color[HTML]{343434} castanho amarelado} & {\color[HTML]{343434} castanho escuro}   \\ \hline
{\color[HTML]{343434} \textbf{tamanho}}     & {\color[HTML]{343434} 29cm}              & {\color[HTML]{343434} 25cm}               & {\color[HTML]{343434} 34cm}              \\ \hline
{\color[HTML]{343434} \textbf{peso}}        & {\color[HTML]{343434} 731g}              & {\color[HTML]{343434} 698g}               & {\color[HTML]{343434} 799g}              \\ \hline
\end{tabular}
\caption{Características diferenciadoras entre subespécies de Suricata.}
\end{table}


\end{document}