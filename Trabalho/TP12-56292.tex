\documentclass[11pt, a4paper]{article}
\usepackage[utf8]{inputenc}
\usepackage[portuguese]{babel}
\usepackage{color}
\usepackage{multicol}
\setlength{\columnseprule}{0.4pt}
\setlength{\columnsep}{2em}
\usepackage{graphicx}
\usepackage{ulem}
\usepackage{makecell}
\title{\huge Seguros de Saúde\\ \vspace{0.3cm} \Large Produç\~ao de Documentos Técnicos}
\usepackage[labelsep=period]{caption}
\captionsetup[table]{name=Tabela}
\renewcommand{\thetable}{\Roman{table}}
\captionsetup[figure]{name=Gráfico}
\renewcommand{\thefigure}{\Roman{figure}}
\author{João Ricardo Silva Matos}
\date{\today}
\usepackage[table,xcdraw]{xcolor}
\usepackage{url}
\usepackage{geometry}
 \geometry{
 a4paper,
 top=20mm,
 bottom=20mm,
 }
\usepackage{tabu}
\begin{document}

\maketitle
\thispagestyle{empty}

\begin{figure} [h]
\centering
\includegraphics[scale=0.5]{seguro-saude.jpg}
\end{figure}
\vspace*{\fill}
Engenharia Informática

TP12 Nº56292

fc56292@alunos.fc.ul.pt

\clearpage
\section{Índice}
\setcounter{page}{1}
\tableofcontents
\clearpage

\section{Introduç\~ao}
Um seguro de saúde tem o objetivo de fornecer o reembolso das despesas médicas, resultado de acidentes ou doenças, dentro dos limites estabelecidos no contrato feito pelo segurado titular e seus dependentes, sendo que o segurado tem livre escolha dos serviços médicos. Os seguros de saúde variam nas suas coberturas, mas geralmente pode abranger Assistência Médica, Urgências, Hospitalização, Ambulatório, entre outros.
\\
\\Diferencia-se do seguro de vida pois este garante ao beneficiário ou ao próprio segurado um pagamento ou renda determinada no caso de morte, ou no caso do segurado sobreviver em um prazo convencionado.
\\
\\Em Portugal, seguros de saúde n\~ao têm a mesma proeminencia que em outros países, principalmente devido á existencia de um sistema nacional de saúde e ao preço relativamente elevado das suas mensalidades.
\\
\\
\begin{multicols*}{2}
[Abaixo encontram-se duas colunas com um pequeno resumo das vantagens e desvantagens dos seguros de saúde.]
1 - Liberdade de escolha.
Um seguro de saúde é uma entrada para um serviço generalizado de cuidados de saúde. Quem beneficia de um seguro de saúde está menos dependente dos serviços médicos públicos, já que tem acesso a redes privadas de cuidados de saúde, onde existem médicos, hospitais, clínicas e centros de diagnósticos disponíveis para o atender.
\\
\\2 - Adaptado ao seu caso.
Como os seguros de saúde são compostos por diferentes coberturas, tem liberdade para escolher a proposta que melhor se adequa às suas necessidades entre as várias ofertas das seguradoras.
\\
\\3 - Tempo de espera.
Pessoas com seguros de saúde beneficiam de rapidez na marcação de consultas ou internamentos, quando comparados com os utentes do serviço público.
\\
\\4 - Descontos para a família.
Todo o agregado familiar pode ser incluído nos seguros de saúde com desconto para as famílias mais numerosas.
\\
\\5 - Ofertas extra.
Os seguros de saúde têm benefícios extra, nomeadamente apoio ao domicilio, medicina preventiva, descontos em parceiros de lazer e bem-estar, entre outros
\\
\\1 - Custo.
É a maior desvantagem dos seguros de saúde. O preço de um seguro de saúde pode variar entre 10 e centenas de euros por mês, dependendo das coberturas desejadas.
\\
\\2 - Carência.
Por vezes poderão ocorrer períodos de tempo em que o seguro não é ativado, conhecidos como períodos de carência. Os seguros de saúde exigem também um período em que o cliente está a pagar mas sem poder usufruir das coberturas.
\\
\\3 - Franquia.
Beneficiários de seguros de saúde poderão por vezes ter de pagar pelos serviços de que usufruíram, apesar de estes estarem previstos no contrato. Isto é bastante desencorajador, já que o beneficiário paga mensalmente o seguro e ainda tem que suportar o custo dos serviços dos quais beneficiou.
\\
\\4 - Co-pagamento.
Existe co-pagamentos sempre que utiliza o seguro de saúde. Um co-pagamento é a percentagem ou a parte do custo do serviço que fica a cargo da pessoa segura. Por vezes estes valores são significativos e reduzem as vantagens do seguro.
\\
\\5 - Exclusões.
Muitas das doenças não estão cobertas, bem como doenças já existentes são excluídas.
\end{multicols*}
\clearpage

\section{Caso de Estudo}

Vamos agora analisar qual o plano de seguro mais vantajoso para o senhor Nelson através de uma comparaç\~ao entre os 3 planos considerados, tendo em conta que o cliente pretende utilizar o seguro de saúde 70\% em horário diurno e 30\% em horário noturno. O gráfico \ref{Fig:1}, bem como  as tabelas \ref{Tab:Seg1} e \ref{Tab:Seg2} foram preenchidas com o número de aluno 56292.

\begin{table}[h]
\centering
\taburulecolor[HTML]{00CD5C}
\begin{tabu}{|c|c|c|c|}
\hline
{\color[HTML]{00CD5C} }                                         & {\color[HTML]{00CD5C} \makecell{Seguro \\ A}} & {\color[HTML]{00CD5C} \makecell{Seguro \\ B}} & {\color[HTML]{00CD5C} \makecell{Seguro \\ C}} \\ \hline
{\color[HTML]{006CAD}  \makecell{Assinatura \\ anual}}                         & {\color[HTML]{006CAD} 29}       & {\color[HTML]{006CAD} 92}       & {\color[HTML]{006CAD} 56}       \\ \hline
{\color[HTML]{006CAD} \makecell{Tarifa de ato \\ médico diurno}   }           & {\color[HTML]{006CAD} 6}        & {\color[HTML]{006CAD} 4}        & {\color[HTML]{006CAD} 3}        \\ \hline
{\color[HTML]{006CAD} \makecell{Tarifa de ato \\ médico noturno} }            & {\color[HTML]{006CAD} 4}       & {\color[HTML]{006CAD} 12}       & {\color[HTML]{006CAD} 7}        \\ \hline
{\color[HTML]{006CAD} \makecell{Atos médicos \\ oferecidos com \\ a assinatura}} & {\color[HTML]{006CAD} 25}       & {\color[HTML]{006CAD} 15}       & {\color[HTML]{006CAD} 10}       \\ \hline
\end{tabu}
\caption{Valores de partida dos diferentes seguros}
\label{Tab:Seg1}
\end{table}

\begin{table}[h]
\centering
\taburulecolor[HTML]{00CD5C}
\begin{tabu}{|c|c|c|c|}
\hline
{\color[HTML]{00CD5C} \makecell{Nº Atos \\ Médicos \\ Anuais}} & {\color[HTML]{00CD5C} \makecell{Seguro \\ A}} & {\color[HTML]{00CD5C} \makecell{Seguro \\ B}} & {\color[HTML]{00CD5C} \makecell{Seguro \\ C}} \\ \hline
{\color[HTML]{006CAD} 10}                     & {\color[HTML]{006CAD} 29}       & {\color[HTML]{006CAD} 92}       & {\color[HTML]{006CAD} 56}       \\ \hline
{\color[HTML]{006CAD} 20}                     & {\color[HTML]{006CAD} 29}       & {\color[HTML]{006CAD} 124}      & {\color[HTML]{006CAD} 98}      \\ \hline
{\color[HTML]{006CAD} 30}                     & {\color[HTML]{006CAD} 56}      & {\color[HTML]{006CAD} 188}      & {\color[HTML]{006CAD} 140}      \\ \hline
{\color[HTML]{006CAD} 40}                     & {\color[HTML]{006CAD} 110}      & {\color[HTML]{006CAD} 252}      & {\color[HTML]{006CAD} 182}      \\ \hline
{\color[HTML]{006CAD} 50}                     & {\color[HTML]{006CAD} 164}      & {\color[HTML]{006CAD} 316}      & {\color[HTML]{006CAD} 224}      \\ \hline
{\color[HTML]{006CAD} 60}                     & {\color[HTML]{006CAD} 218}      & {\color[HTML]{006CAD} 380}      & {\color[HTML]{006CAD} 266}      \\ \hline
{\color[HTML]{006CAD} 70}                     & {\color[HTML]{006CAD} 272}      & {\color[HTML]{006CAD} 444}      & {\color[HTML]{006CAD} 308}      \\ \hline
{\color[HTML]{006CAD} 80}                     & {\color[HTML]{006CAD} 326}      & {\color[HTML]{006CAD} 508}      & {\color[HTML]{006CAD} 350}      \\ \hline
{\color[HTML]{006CAD} 90}                     & {\color[HTML]{006CAD} 380}      & {\color[HTML]{006CAD} 572}      & {\color[HTML]{006CAD} 392}      \\ \hline
{\color[HTML]{006CAD} 100}                    & {\color[HTML]{006CAD} 434}      & {\color[HTML]{006CAD} 636}      & {\color[HTML]{006CAD} 434}      \\ \hline
{\color[HTML]{006CAD} 110}                    & {\color[HTML]{006CAD} 488}      & {\color[HTML]{006CAD} 700}     & {\color[HTML]{006CAD} 476}      \\ \hline
{\color[HTML]{006CAD} 120}                    & {\color[HTML]{006CAD} 542}      & {\color[HTML]{006CAD} 764}     & {\color[HTML]{006CAD} 518}      \\ \hline
\end{tabu}
\caption{Preço de atos médicos}
\label{Tab:Seg2}
\end{table}

\begin{figure} [h]
\centering
\includegraphics[scale=0.7]{Sem Título.png}
\caption{Comparação do preço de atos médicos por seguro}
\label{Fig:1}
\end{figure}


\clearpage
\section{Conclus\~ao}
Conclui-se, com base nas tabelas \ref{Tab:Seg1} e \ref{Tab:Seg2} e no gráfico \ref{Fig:1} que, caso pretenda fazer 100 ou  mais atos médicos anuais, seria ideal optar pelo seguro C, visto que este oferece a melhor tarifa de ato médico. Caso  preveja fazer menos de 100 atos médicos por ano a melhor escolha será o seguro A, dado que o seu preço de assinatura anual é o mais baixo.

\clearpage
\section{Bibliografia}
\begin{thebibliography}{9}
\bibitem{latexcompanion} \textit{Seguro de Saúde – Vantagens e desvantagens, Reorganiza}\\ \url{https://reorganiza.pt/seguros-de-saude-vantagens-desvantagens/}
\bibitem{latexcompanion} \textit{Seguro, Wikipédia}\\
\url{https://pt.wikipedia.org/wiki/Seguro#Classifica%C3%A7%C3%A3o_dos_seguros}
\bibitem{latexcompanion} \textit{Seguro de Saúde – Vale a Pena?}\\ \url{https://www.doutorfinancas.pt/seguros/seguro-de-saude/seguro-de-saude-vale-a-pena/}
\bibitem{latexcompanion} \textit{7 motivos para fazer um seguro de saúde}\\ \url{https://www.bancoctt.pt/os-seus-seguros/7-motivos-para-fazer-um-seguro-de-saude.html}
\end{thebibliography}


\end{document}