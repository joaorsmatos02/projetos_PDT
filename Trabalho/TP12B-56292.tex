\documentclass[11pt, xcolor=table]{beamer}
\usepackage[utf8]{inputenc}
\usepackage[portuguese]{babel}
\usepackage{color}
\usepackage{ulem}
\usetheme{AnnArbor}
\usecolortheme{dolphin}
\definecolor{Color}{RGB}{204,204,255}
\setbeamercolor{background canvas}{bg=Color}
\author{João Ricardo Silva Matos}
\title{Seguros de Saúde\\ Produç\~ao de Documentos Técnicos}
\usepackage[labelsep=period]{caption}
\captionsetup[table]{name=Tabela}
\renewcommand{\thetable}{\Roman{table}}
\captionsetup[figure]{name=Figura}
\renewcommand{\thefigure}{\Roman{figure}}
\usepackage{tabu}
\usepackage[table]{xcolor}
\setbeamertemplate{caption}[numbered]
\newcommand\MC[1]{\begin{tabular}{@{} >{\color[HTML]{00CD5C}}c @{} } #1 \end{tabular}}

\begin{document}

\maketitle

\clearpage

\begin{frame}
\frametitle{Preço por quantidade de atos médicos}
    \begin{table}
    \small
    \arrayrulecolor[HTML]{00CD5C}
\begin{tabular}{|*{4}{>{\color[HTML]{006CAD}}c|} }
    \hline
\MC{Nº Atos \\ Médicos \\ Anuais}
    &   \MC{Seguro \\ A}
        &   \MC{Seguro \\ B}
            &   \MC{Seguro \\ C}    \\
    \hline
10  & 29  & 92  & 56  \\ \hline
20  & 29  & 124 & 98 \\ \hline
30  & 56 & 188 & 140 \\ \hline
40  & 110 & 252 & 182 \\ \hline
50  & 164 & 316 & 224 \\ \hline
60  & 218 & 380 & 266 \\ \hline
70  & 272 & 444 & 308 \\ \hline
80  & 326 & 508 & 350 \\ \hline
90  & 380 & 572 & 392 \\ \hline
100 & 434 & 636 & 434 \\ \hline
110 & 488 & 700 & 476 \\ \hline
120 & 542 & 764 & 518 \\ \hline
\end{tabular}
\caption{Preço de atos médicos}
\label{Tab:1}
    \end{table}
\end{frame}


\clearpage

\begin{frame}
\frametitle{Preço por quantidade de atos médicos}
\begin{figure}
\includegraphics[scale=0.5]{Sem Título.png}
\caption{Comparação do preço de atos médicos por seguro}
\label{Fig:1}
\end{figure}
\end{frame}

\clearpage

\begin{frame}
\frametitle{Conclusão}
Com  base nos dados acima referidos podemos concluir que o seguro mais vantajoso será:
\vspace{1cm}
\begin{enumerate}
\item O seguro C, caso preveja a realização de mais de 100 atos médicos por ano.
\item O seguro A, caso seja previsível a realização de 100 ou menos atos médicos por ano.
\end{enumerate}
\end{frame}

\clearpage

\begin{frame}
\frametitle{Obrigado pela sua atenção}
\begin{figure}
\includegraphics[scale=0.35]{seguro-saude (1).jpg}
\caption{Seguros de saúde}
\label{Fig:2}
\end{figure}
Fonte:https://www.acarteira.pt/artigos-seguros/seguro-de-saude-ou-plano-de-saude-como-escolher/
\end{frame}

\end{document}